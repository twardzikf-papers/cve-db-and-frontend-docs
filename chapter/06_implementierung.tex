\chapter{Implementierung (Filip)}

\section{Backend}
Das Backend wurde mit der Hilfe von Spring Framework erstellt. Der Webserver stellt dem Benutzer die entsprechenden Daten zur Verfügung. Außerdem realisiert er die Kommunikation mit der Datenbank, verschiedene  Such verfahren als auch das Herunterladen und Einspeisen der neuen Daten in die Datenbank. Diese Funktionalitäten wurden in vier Klassen implementiert.\\

\textit{Application} ist verantwortlich für den Start des Servers und falls eine entsprechend Flag übergeben wurde, für das Herunterladen und Einspeisen des neuen Datensatzes. Dieses kann auch optional  als automatische Tätigkeit, die in festen Zeitintervallen geplant ist, implementiert werden. In dieser Implementierung wurde es jedoch auf die manuelle, Flag-gesteuerte Weise implementiert.\\

\textit{MongoDBImport} ist eine Klasse, die den eigentlichen Arbeitsablauf bei dem Herunterladen und Einspeisen der neuen Datensätze implementiert. Der neueste Feed wird zuerst heruntergeladen. Danach wird er entpackt und mit Hilfe eines externen Python Skripts entsprechend formatiert, damit die Daten vollständig von der Datenbank gelesen werden können. Schließlich werden die Daten mittels Ausführung der Update-Funktion in die Datenbank eingelesen.\\

\textit{SearchController} ist zuständig für die Behandlung von Views und API Anfragen. Dabei sind die zwei API POST Anfragen, \textit{search/submit} und \textit{search/page}, von entscheidender Bedeutung. Sie sind verantwortlich für die Ausführung der Suchanfrage und für die Berechnung der resultierenden Daten auf der Backend-Seite. Eine derartige Teilung der Funktionalität realisiert Pagination auf der Backend-Seite. Auf diese Weise können bestimmte Teile der Ergebnisse auf Nachfrage nachgeliefert werden.\\

\textit{SearchMethods} realisiert vor allem die Verbindung mit der Datenbank und die eigentlichen Suchverfahren - den Kern der Funktionalität der Applikation. Die Suchen nach CVE, CPE, CWE, CVSS und die Textsuche werden zuerst als einzelne Anfragen generiert und erst vor der Ausführung in der Datenbank verundet. Diese Lösung resultiert in einer intuitiver Suche aus der Sicht des Benutzers, bei der er mit weiteren Kriterien die Anzahl der Ergebnisse flexibel begrenzen kann. Außerdem führt die Klasse eine Möglichkeit ein zwischen einer lokalen und und externen Datenbank zu unterscheiden. Während der Entwicklung wurde ausschließlich eine lokale Datenbank verwendet, in der Testphase jedoch wurde auch eine externe Datenbank benutzt, um die Leistung zu testen.\\

\begin{java}
public class Application {
    public static void main(String[] args);
}
public class MongoDBImport {
    public static void downloadJSON();
    public static void copy(InputStream input, OutputStream output, int bufferSize);
    public static void unzip(String zipfile);
    public static void mongoImport();
}
public class SearchController implements WebMvcConfigurer {
    public void addViewControllers(ViewControllerRegistry registry);
    public JSONObject searchSubmit(@RequestParam Map<String, String> params);
    public JSONObject searchPages(@RequestParam Map<String, String> params);
}
public class SearchMethods {
    public static void setLocalDatabaseFlag();
    public static boolean getLocalDatabaseFlag();
    public static void dbConnect();
    public static JSONObject mainSearch(Map<String, String> params);
    public static JSONObject getPage(Map<String, String> params);
}
\end{java}

\section{Frontend}
Auf der Clientseite hat sich das Team dazu entschlossen, eine Webpräsenz als Single Page Application zu implementieren. Die Seite besteht aus einer HTML und einer CSS Datei, in der die Struktur und der statische Teil der Darstellung definiert sind und aus einer Javascript Datei, die die ganze Funktionalität der Seite realisiert.\\

% HTML & CSS - Struktur der Webseite
Die Struktur der Webseite besteht aus drei Inhaltsbereichen. Links angeordnet ist ein  Menü mit Referenzen auf  statische Unterseiten, die die Dokumentation sowohl der API als auch des Projektes als Ganzes beinhalten. Unter den Referenzen befindet sich das Formular für die Suchanfrage mit dem direkt in HTML als \textit{pattern} Attribut verfügbaren  Validierungsverfahren. Die Ergebnisse einer Suchanfrage erscheinen in dem rechts angeordneten Inhaltsbereich. \\

% JS - Verhalten der Webseite 
Die Webseite realisiert folgende Funktionen: dynamisches Umschalten der Unterseiten, Aufnehmen der Suchanfragen, Kommunikation mit dem Server mittels API und geeignete Darstellung der Ergebnisse mit Pagination. Sowohl das Umschalten, Annehmen der Suchanfragen als auch die Kommunikation mittels API wurden mit der Hilfe von \textit{event listeners} und \textit{ajax requests} aus der \textit{jQuery} Bibliothek umgesetzt. Die empfangene Ergebnisse beinhalten immer die ganzen Datensätze, deshalb sind die Parserfunktionen von zentraler Bedeutung um nur die relevante Attribute zu extrahieren. Die  Funktionen verfolgen das Bennenungsschema \textit{get$\langle$NameOfAttribute$\rangle$} und die grobe Struktur: Behandlung der Randfälle wie beispielsweise Nullwerte. Alle geben auch eine HTML-Zeichenkette zurück, die direkt als Inhalt einer Tabellenzelle hinzugefügt werden kann. Die Parserfunktionen werden von der Überfunktion benutzt, die für die Kommunikation mit dem Server, Erstellung der Metriken und Generierung der Tabelle mit Ergebnissen zuständig ist. Die Tabelle zeigt die wichtigsten Attribute der Datensätze (CVE, Vendors, Description, CPE, CVSS Score) sofort als sichtbaren Inhalt. Weitere Informationen zum jeweiligen Datensatz, können durch den erweiterbarer Tabelleneintrag abgerufen werden.\\

\section{Tests}
Ausgewählte Kernfunktionen sowohl von dem Server als auch von der Webseite wurden mittels Unit Tests auf Korrektheit geprüft. Mit der Hilfe von \textit{JUnit} Tests \cite{Massol2004JUnitAction} wurde getestet, ob das Herunterladen und das Entpacken der JSON Dateien fehlerfrei abläuft. \textit{Jest} Javascript Bibliothek wurde verwendet, um die Korrektheit des Inhalts und der Darstellungsform der einzelnen extrahierten Informationen zu prüfen \cite{Banker:2011:MA:2207997}.

