\begingroup
\renewcommand{\cleardoublepage}{}
\renewcommand{\clearpage}{}
\chapter{Problembeschreibung (Giulia)}\label{chap:ack}
\endgroup
%\chapter{Problembeschreibung (Giulia)}
In den letzten Jahren wurden bereits riesige Mengen an Daten zu Schwachstellen in der National Vulnerability Database (NVD) \cite{nvd} gesammelt. Ziel dieser Arbeit ist es diese Masse an Daten strukturiert, effizient und benutzerfreundlich darzustellen und eine einfache, schnelle Suche zu ermöglichen. Hierzu sollen die Daten als JSON- oder XML-Feed aus der NVD \cite{nvd} ausgelesen, verarbeitet und so dargestellt werden, dass eine effiziente Suche gewährleistet ist. Dies beinhaltet zudem eine lokale Speicherung der Informationen in einem entsprechenden Modell und damit einhergehend ein automatisiertes Update der lokalen Datenbank, um diese möglichst aktuell zu halten. Eine ansprechende, intuitiv bedienbare Benutzeroberfläche mit Verwaltungs- und Steuerfunktionen soll eine vielseitige und komplexe Suche ermöglichen. Hierzu gehört unter Anderem die Suche nach einem beliebigen Text, als auch nach wichtigen Kennzahlen, wie beispielsweise CPE, CWE und CVSS. Die gefundenen Daten sollen schließlich mittels REST-API dargestellt werden. Zur Überprüfung der Effizienz und Korrektheit der Suche sollen darüberhinaus die Anzahl der gefundenen Einträge und die Dauer der Suchanfrage angezeigt werden. 



