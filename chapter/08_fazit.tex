\chapter{Fazit und Ausblick (Patrick)}
    
    % https://link.springer.com/chapter/10.1007/978-3-642-23088-2_15
    Aufgrund der klar formulierten und ausführlichen Anforderungen gestaltete sich die Analyse der Anforderungen und die Umsetzung des Projektes ohne Probleme. Die Erfahrung, die mit JHipster gesammelt wurde, konnte zügig mit Spring umgesetzt werden, sodass mit Hilfe von \glqq Model View Control \grqq{} eine funktionale Applikation entwickelt werden konnte.
    \\
    
    Sowohl die Einteilung der Arbeitspakete, als auch die zeitliche Planung führten zu einem Resultat, welches die Anforderungen vollständig erfüllt. Durch wöchentliche Treffen und die Einhaltung von SCRUM, verbesserten sich die Absprachen und die Zusammenarbeit. Sämtliche Aufgaben wurden in geplanter Zeit erledigt und Entscheidungen in gemeinsamer Abstimmung getroffen. Aufgrund der guten Zusammenarbeit, welche gegenseitige Kritik, Anmerkungen, Feedback und Verbesserungsvorschläge beinhaltet, sind auch weiterführende Projekte denkbar. Beispielsweise die Verwendung der Suche über eine REST Schnittstelle für einen Vergleich zwischen einer Software-Datenbank und der NVD-Datenbank. Ein Vergleich könnte Schwachstellen bestehender System entdecken und mit Warnungen auf Updates für mehr Sicherheit in IT-Systemen hinweisen. 
    \\  
    
    Für zukünftige Projekte wie z.B. das Projekt von Rainer Bye et. al.\cite{bye:2010:CollSec} kann die Applikation genutzt werden um vorhandene Software-/Hardware Komponenten in IT-Netzwerken zu analysieren und mit aktuellen Einträgen aus der NVD zu vergleichen.
    Ein solcher Vergleich der Einträge kann dabei automatisiert werden, beispielsweise mit Nagios\cite{Nagios}, einer IT Infrastruktur Monitoring Applikation. Mit dieser Applikation können Benachrichtigungen und Warnungen eingerichtet werden um vor eventuellen Gefahren gewarnt werden und schaden abgewendet.
