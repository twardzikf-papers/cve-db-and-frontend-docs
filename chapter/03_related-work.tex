\chapter{Related Work (David)}
Wie aktuelle Statistiken vom Bundesamt für Informationssicherheit (BSI) zeigen, nimmt die Anzahl der Angriffe gegen IT-Systeme ständig zu \cite{furSicherheitinderInformationstechnikDie2017}. Dementsprechend gewannen auch Plattformen beziehungsweise Datenbanken, die sicherheitsrelevante Informationen zu aktuellen Bedrohungen und Handlungsempfehlungen zur Verfügung stellen an Bedeutung. Einrichtungen, wie zum Beispiel das dem \glqq US-CERT\grqq{}  zugehörigen \glqq National Cybersecurity and Communications Integration Center\grqq{} (NCCIC) \cite{NationalSecurity} informiert mit Hilfe von Warnungen, Analysen und Berichten über die aktuelle Gefährdungslage und stellt Hinweise und Ratschläge für Betroffene zur Verfügung. Eine vergleichbare Rolle hat das BSI in Deutschland.
\\

Die genannten Einrichtungen referenzieren dabei auf konkrete Schwachstellen, die durch CVE-Kennungen eindeutig identifizierbar sind. CVE steht hierbei für \glqq Common Vulnerabilities and Exposures\grqq{}. Es handelt sich hierbei um ein Industriestandard, dessen Ziel die Einführung einer einheitlichen Namenskonvention für Schwachstellen in Computersystemen ist \cite{Mell2006CommonSystemb}. Neben CVE sind weitere Standards entstanden, um Eigenschaften beziehungsweise Merkmale von Schwachstellen eindeutig beschreiben zu können. So kann beispielsweise der Schweregrad einer möglichen oder tatsächlichen Schwachstelle durch den \glqq Common Vulnerability Scoring System\grqq{} (CVSS) Standard bewertet werden  \cite{Mell2006CommonSystemb}. Neben dem weit verbreiteten CVSS Standard existiert auch \glqq Common Weakness Scoring System\grqq{} (CWSS), der ebenfalls zur Priorisierung und Bewertung von Sicherheitslücken eingesetzt werden kann. Ein weiterer Standard der im diesem Kontext genannt werden muss, ist \glqq Common Weakness Enumeration\grqq{} (CWE). Mit diesem werden verschiedene Typen beziehungsweise Varianten von Schwachstellen definiert \cite{CWECWSS}.
\\

Es ist offensichtlich, dass der kombinierte Einsatz dieser Standards eine Möglichkeit schafft, um Schwachstellen und ihre Eigenschaften, sowie betroffene Systeme eindeutig zu beschreiben und zu einem einheitlichen Datensatz zu vereinen. Aus diesem Potenzial heraus ist die \glqq National Vulnerability Database\grqq{} (NVD) entstanden. Die vom \glqq National Institute of Standards and Technology\grqq{} (NIST) gegründete Plattform ist eine der am weit verbreitetsten Datenbanken zur Sammlung von Schwachstellen-Informationen. Die zuvor genannten Standards werden hier eingesetzt, um eine möglichst hohe Qualität zu erreichen \cite{Zhang2011AnVulnerabilities}. Unter dieser Qualität ist zu verstehen, dass Schwachstelleninformaionen sorgfältig kategorisiert und bewertet werden und diese dann aufgrund der Standardisierung nach unterschiedlichen Kreterien durchsuchbar werden. Angereichert durch beispielsweise Kommentare der betroffenen Hersteller kann ein umfangreicher und aussagekräftiger Datensatz zu einer Schwachstelle entstehen. Zusätzlich stellt die NVD eine Feed-Funktionalität zur Verfügung, wodurch Nutzer Zugriff auf aktuelle Datensätze erhalten.
\\

Auf Grundlage dieser standardisierten und umfangreichen Datensätze lassen sich neue Potentiale erschließen. So befassen sich verschiedene Forschungsarbeiten mit der Vorhersage von Bedrohungslagen beziehungsweise Cyber-Risiken \cite{Zhang2015PredictingDatabase}, bis hin zur Prognose von Schwachstellen in konkreten Systemen und Anwendungen \cite{Zhang2011AnVulnerabilities}.
