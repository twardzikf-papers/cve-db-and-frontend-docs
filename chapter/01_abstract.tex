\chapter*{Abstract (Patrick)}
% Folgende Infos aus dem Leitfaden für wissenschaftliche  Arbeiten der TU Berlin:
% https://www.proscience.tu-berlin.de/fileadmin/fg15/archiv_berichte_reporting/2013/assisthesis_studierendenversion1.pdf
% Kurzfassung, Zusammenfassung: 
% 1.) Das Abstract verrät mir, ob die Arbeit für mich interessant ist.
% 2.) Das Abstract ist allein verständlich, ohne die gesamte Arbeit zu lesen.
% 3.) Der Aufbau gleicht dem des Originaltextes:
% Zitat ASSIS Thesis: 
% „Auskunft über das behandelte Gebiet, Zielsetzungen, Hypothesen, Methoden, Ergebnisse und Schlußfolgerungen der im Originaldokument enthaltenen Überlegungen und Darstellungen, einschließlich der Fakten und Daten. [DIN 1426, S. 3]
Die Relevanz neuer Sicherheitsmechanismen in modernen IT-Infrastruktur Systemen steht immer öfter im Mittelpunkt, wie Angriffe auf große Unternehmen in den letzten Jahren gezeigt haben. Die National Vulnerability Database (NVD) stellt eine Datenbank zur Verfügung, die sicherheitsrelevante Informationen zu IT-Bezogenen Schwachstellen bereit stellt. Ziel der vorliegenden Arbeit ist es, diese Informationen übersichtlich, benutzerfreundlich aufzubereiten und zur Verfügung zu stellen. Es stellt sich heraus, dass die Umsetzung des Projektes mit Spring und MongoDB eine optimale Methode bietet, dem Entwurfsmuster Model View Control (MVC) nachzukommen. Mit Projektmanagement-Methoden wie die Nutzung eines Gantt-Diagramms und der Anwendung von SCRUM für die agile Softwareentwicklung ist es gelungen, die Informationen der NVD aufzubereiten und das Projekt in der geplanten Zeit umzusetzen. Die Planung und Entwicklung innerhalb des Teams und wöchentliche Meetings führen zu einem Ergebnis, dass sämtliche Anforderungen erfüllt und für zukünftige Projekte, wie beispielsweise Applikationen zur Überwachung von IT-Netzwerken, eingesetzt werden kann.