\chapter{Motivation (David)}
Software ist im heutigen digitalen Zeitalter das Fundament, auf dem die Wirtschaft und mittlerweile auch die Gesellschaft basiert. Immer umfangreichere Software-Lösungen finden in der Industrie, aber auch in kapitalintensiven und sicherheitskritischen Bereichen Anwendung \cite{Belli1998MethodenSoftware}. Die Komplexität die bei der Entwicklung derartiger Systeme bewältigt werden muss, stellt zunehmend eine Herausforderung für das Projekt- und Entwicklungsmanagement dar. Die einhergehende Fehleranfälligkeit von komplexen Software-Systemen kann dabei ein hohes Gefahrenpotenzial implizieren. Je nach dem, wo die Software eingesetzt wird, können Software-Fehler neben großen wirtschaftlichen Schäden (zum Beispiel im Bereich des Bankwesens) auch zu Gefahren für Leib und Leben von Menschen führen. So ist die Zuverlässigkeit und Korrektheit von in Passagierflugzeuge eingesetzter Software essentiell für einen sicheren Transport der Passagiere.
\\

Im Allgemeinen können vorhandene Softwarefehler unter bestimmten Bedingungen, aber auch durch bewusstes Verursachen (Provozieren der verantwortlichen Bedingungen) durch Dritte ausgelöst werden. In diesem Kontext tritt das Wort \glqq Schwachstelle\grqq{} (oder auch \glqq Sicherheitslücke\grqq{}) in Erscheinung. Beide Begriffe definieren den Zustand eines Systems, welches durch einen bestimmten Angriff manipuliert oder beschädigt werden kann. Unter dem Begriff Angriff ist in diesem Kontext das mutwillige Ausnutzen beziehungsweise Auslösen eines Software-Fehlers zu verstehen \cite{Al-Fedaghi2010System-basedVulnerability}. Auf diese Weise können beispielsweise Authentifizierungs- und Authorisierungsmechanismen umgangen werden, um sich unberechtigten Zugang zu fremden Computer-Systemen und Netzwerken zu verschaffen. Die Erkennung und Verhinderung von derartigen Angriffen wird in vielen Forschungsarbeiten thematisiert, wie zum Beispiel in dem Buch von Viega und McGraw zur Vermeidung von Sicherheitsproblemen während der Entwicklung neuer Software \cite{Viega2001BuildingWay}.
\\

Da aber auch aktuelle und weit verbreitete Software-Systeme immer wieder von Schwachstellen betroffen sind, ist eine schnelle Behebung dieser, sowie die sofortige Aufklärung der Nutzer enorm wichtig. Für diesen Zweck existieren viele verschiedene Plattformen beziehungsweise Datenbanken, die Informationen zu aktuellen beziehungsweise neu entdeckten Schwachstellen standardisiert speichern und so betroffene Personen und Unternehmen einen schnellen Zugriff zu für sie relevante Daten ermöglichen \cite{Mell2006CommonSystem}.

%Der folgende Projektbericht stellt die Umsetzung einer Plattform dar, auf der Informationen zu Schwachstellen speichert und diese durch verschiedene Such- und Filterfunktionalitäten zur Verfügung stellt. Dabei werden alle Informationen von der National Vulnerability Database (NVD) importiert.
