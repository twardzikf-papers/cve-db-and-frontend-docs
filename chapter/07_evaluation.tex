\chapter{Evaluation (Patrick)}

    Zu Beginn des Projektes hat das Team sich dazu entschlossen das Java-Framework JHipster zu benutzen \cite{JHipsterApplications}. In Verbindung mit einem Klassendiagramm, passend zu der Datenstruktur der NVD Datenbank, sollten die Daten der National Vulnerability Database (NVD)\cite{nvd}, in eine relationale Datenbank gespeichert werden. Aufgrund der Unverhältnismäßigkeit zwischen Aufwand und Nutzen ergab eine erneute Evaluation, dass das Projekt mit dem Spring Framework und noSQL (MongoDB \cite{mongodb}) einfacher und effektiver umgesetzt werden kann. Diese Entscheidung ermöglichte ein effektiveres und produktiveres Arbeiten mit klar abgegrenzten Aufgaben (SCRUM). Mehrere Methoden wie z.B.: Feedback Gespräche, Reviews und Sprint Planungen ermöglichten eine effiziente Bearbeitung der Anforderungen. 
    \\

    Das Resultat des Projektes ist eine Webapplikation in der nach Software-Schwachstellen, die in der NVD gelistet sind, gesucht werden kann. Die Ergebnisse der Suche werden in einer Liste mit aufklappbaren Elementen dargestellt. Eine farbliche Markierung, angelehnt an das \glqq Common Vulnerability Scoring System\grqq{} (CVSS) bietet eine Unterscheidung der Ergebnisse nach Ihrer \glqq Schwere\grqq{} \cite{CommonSIG}. 
    Die so entstandene Plattform, siehe Abb. A.1, ermöglicht einen schnellen und übersichtlichen Zugriff auf die Daten der NVD.
    Die Umsetzung der Suche mit Regex ermöglicht eine anpassbare Applikation, die mit geringen Modifikationen verschiedene Suchmöglichkeiten bietet. MongoDB in Verbindung mit noSQL speichert zwar redundante Daten wie Beispielsweise den Hersteller, jedoch haben alle (manuell) getesteten Suchanfragen in angemessener Zeit Resultate geliefert. Die Umsetzung der 'Datenstruktur' erforderte kein aufwändiges relationales Datenmodell wie es beispielsweise bei Nutzung mit SQL der Fall wäre. Da jedes Datenbank update durchschnittlich 8-12 MegaByte groß ist, ist eine Umsetzung mit noSQL kein Problem. 
    \\
    
    Die Anwendung der SCRUM-Methode für die Realisierung des Projektes hat in der frühen Projektphase dazu geführt, dass wir von JHipster zu einem Spring Projekt gewechselt haben \cite{Rising2000TheTeams}. Dieser Wechsel beschleunigte das Erreichen der Meilensteine um ein vielfaches, sodass das Projekt zügig, mit allen Anforderungen realisiert werden konnte.
    \newpage